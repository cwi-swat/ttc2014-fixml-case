\documentclass[submission,copyright,creativecommons]{eptcs}
\providecommand{\event}{TTC 2014} % Name of the event you are submitting to
\usepackage{breakurl}             % Not needed if you use pdflatex only.
\usepackage{rascal}
\usepackage[T1]{fontenc}
\usepackage[scaled=0.8]{beramono}
\title{The TTC 2014 Movie Database Case: Rascal Solution\thanks{This
    research was supported by the Netherlands Organisation for
    Scientific Research (NWO) Jacquard Grant ``Next Generation
    Auditing: Data-Assurance as a service'' (638.001.214).}}
\author{Pablo Inostroza \and Tijs van der Storm}
\def\authorrunning{Inostroza and Van der Storm}
\def\titlerunning{TTC'14: Rascal}

\begin{document}
\maketitle

\begin{abstract}
Rascal is meta programming language for processing source code in the broad sense (models, documents, formats, languages, etc.). In this short note we discuss the implementation of the `TTC'14 FIXML to Java, C\# and C++ Case'' in Rascal. In particular we will highlight the challenges and benefits of using a functional programming language for ...
\end{abstract}

\section{Introduction}

Rascal is a meta programming language for source code analysis and transformation~\cite{Rascal,RascalGTTSE}. 
Concretely, it is targeted at analyzing and processing any kind of ``source code in the broad sense''; this includes importing, analyzing, transforming, visualizing and generating, models, data files, program code, documentation etc.
 
Although Rascal features a Java-like syntax, it is a functional programming in that all data is immutable (implemented using persistent data structures), and function programming concepts are used throughout: algebraic data types, pattern matching, higher-order functions, comprehensions etc. 

Specifically for the domain of source code manipulation, however, Rascal features powerful primitives for parsing (context-free grammars), traversal (visit statement), relational analysis (transitive closure, image etc.), and code generation (string templates). 
The standard library includes programming language grammars (e.g., Java), IDE integration
with Eclipse, numerous importers (e.g. XML, CSV, YAML, JSON etc.) and a rich visualization framework. 

\subsection{The transformation}

The solution transformation has been broken down into the following sub transformations:

\begin{enumerate}
\item XML text to model of XML metamodel
\item model of XML metamodel to a metamodel of OO programming languages
\item OO metamodel to program text (for different OO programming languages)
\end{enumerate}

We have addressed all these cases as we proceed to explain.

\section{Sub transformation 1: XML text to model of XML metamodel}


\section{Sub transformation 2: XML metamodel to OO metamodel}

We have defined an OO metamodel that suits the purpose of this particular task. This means that It is not necessarily comprehensive enough to model a complex OO system, but it serves as an intermediate model from which all the desired output in this context could be generated. 

The following datatypes capture the structure of the OO metamodel:
%\CAT{Keyword}{alias} Id = \CAT{Keyword}{int};
\begin{rascal}
\CAT{Keyword}{data} OOModel = oomodel(\CAT{Keyword}{list}{}[Class] classes);
\CAT{Keyword}{data} Class = class(\CAT{Keyword}{str} name, \CAT{Keyword}{list}{}[Field] literalFields, \CAT{Keyword}{list}{}[Field] objFields);
\CAT{Keyword}{data} Type = tipe(\CAT{Keyword}{str} className);
\CAT{Keyword}{data} Field = objField(Type tipe, \CAT{Keyword}{str} name, \CAT{Keyword}{str} altName, \CAT{Keyword}{list}{}[Field] vals) 
           | literalField(Type tipe, \CAT{Keyword}{str} name, \CAT{Keyword}{str} altName, \CAT{Keyword}{str} val);
\CAT{Keyword}{data} Ref = ref(\CAT{Keyword}{str} container, \CAT{Keyword}{str} field) | ref(\CAT{Keyword}{str} name);
\CAT{Keyword}{data} Statement = assignment(Ref lhs, Ref rhs);
\end{rascal}



\section{Concluding Remarks}

The tasks were easy to implement, with little code. The size of the
implementation is around 130 SLOC, including some helper functions,
but excluding loading the model from XML which is another 38 SLOC.

Rascal's module system showed its benefits; some tasks could be
implemented as modular extensions of earlier tasks, combining
extension of data types (Extension Task 2) and extension of functions
(Extension Tasks 3 and 4).

Performance: extracting cliques $> 2$ takes longer than 5 minutes on a
3.3Mb IMDB file. The most probable reason is that
\texttt{combinations} is slow, even though we use a dynamic
programming algorithm; it is possible that immutability works against
us here.

\section{Bibliography}

\nocite{*}
\bibliographystyle{eptcs}
\bibliography{generic}
\end{document}
